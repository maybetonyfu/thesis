% Chapter 3

\chapter{Effects of Interactive Type Error Debugging Using Chameleon}
\label{chap:chameleon:eval} 

\section{Research Questions}

The studies of Chameleon aim to address the following research question:

\begin{itemize}
	\item Can visual tools help programmers help find bugs?
	\item Can interactive type error narrowing help programmers help solve type errors?
\end{itemize}

\section{Experiment Design}

\subsection{\textbf{Recruitment}}

Participants were recruited via the Reddit \textit{r/haskell} and \textit{r/programminglanguages} communities. 
%Recruiting from social media allowed us to access a more diverse demographic that better represent the true population of Haskell programmers. 
Participation is fully anonymized; detailed ethical implications of these experiments are reviewed and approved by the IRB of the authors' institution.

\subsection{\textbf{Experiment setting}}
Experiments were conducted online and unsupervised. 
%Participants took the study online via a web browser and at the physical venue of their choosing. 
All user studies use a web-based debugging environment developed by the authors. 
%Conducting the studies online helped us avoid variation when performing tasks in unfamiliar places and using different setups. 


\subsection{\textbf{Training and group assignment}}
After consent, participants received interactive training on the tool interface and interactive features. Participants were also shown a cheat sheet summarizing the key functionality of the interface, and had access to the cheat sheet at all times during the study. Participants were given 4 trial runs (2 for each setting) before the data collection started. 
All the studies used a within-subject design to evaluate the effectiveness of different tools or feature sets while counterbalancing the difference in programming proficiency between participants. In each study, participants were required to complete a series of programming tasks (8 for studies 1a and 1b, 9 for study 2). At each task, a participant receives a single Haskell file that contains one or more type errors. They were then asked to correct the code with the help of the given tool.


\subsection{\textbf{Data Collection}}
Time is measured from the start of each task to the first time the program is successfully type-checked and also passes all the functional tests. Participants are able to skip a task if they are stuck. 
% The data is automatically recorded by the online debugging environment. To not introduce a barrier to completing the study, every task can be skipped if the participant made three failed attempts or is stuck for over 1 minute on the task.
After completing all tasks, participants are prompted to complete a debriefing survey. The survey questions include their Haskell experience and feedback on the tools.

We used a browser session recording tool~\cite{openreplay_openreplay_2022} to record the study sessions. This allows us to identify usability issues in the study and to recognize general patterns. 

\section{Results}

\section{Discussion}
\section{Conclusion}
