

% Chapter 6

\chapter{Conclusion and Future Work}

\label{chap:conclustion} 

\section{Conclusion}

We contribute Chameleon, an interactive Haskell type error debugging tool. Internally, Chameleon computes all relevant locations that contribute to the type of error. Via a set of iteratively designed interface, Chameleon preserves the two alternatives of the type error and the supporting evidence for each.


We also contributed a series of studies of the effects of debugging with visual representation of types and interactively explored type errors. We show that there is a difference between using traditional tools and enhanced type error debugging tools like Chameleon. And we show that this difference is more significant when debugging complex type errors.

 

We contribute Goanna, a Haskell type error debugging tool. Like Chameleon, Goanna iterates relevant locations that contribute to the type error and presents alternatives to the type error. Different from Chameleon, Goanna will exhaust all possible alternative explanations of the type error. Also, Goanna presents a type error by dividing it into a list of potential causes and their respective fixes. With Goanna, Haskell programmers can resolve type errors by exploring a list of potential error root causes. These causes are ordered using our heuristics so that the more likely causes are on top. We show that via our empirical evaluation that Goanna outperforms existing Haskell compilers when explaining the type error, with the slight disadvantage of an increased computation time.


We contribute GeckoGraph, a graphic notation for Haskell types. GeckoGraph describes the same information as a type signature does, but uses colours, shapes, and symbols to make certain structures easy to identify at a glance. GeckoGraph is designed to use visual elements to improve the understanding of type-level concepts. This includes type classes, parametric type variables, and high-rank types. When used to compare two types, GeckoGraph helps clarify differences visually. It makes errors like too few or too many arguments in applications, unmet type class constraints obvious.


We conducted a large-scale study on the effectiveness of using GeckoGraph to perform a series of Haskell tasks. We concluded that with GeckoGraph, programmers are able to succeed in harder tasks.

\section{Future Work}
