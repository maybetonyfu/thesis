

% Chapter 5

\chapter{GeckoGraph: A Diagrammatic Notation for Haskell Types}

\label{chapter5} 


\section{Motivation Example}

Programming with types has been an important practice since the early days of software development. Those who use typed programming languages enjoy the assurance that, when composing code, the type checker will diligently report any errors. In more sophisticated programming contexts, types often serve as a protocol that facilitates communication between the authors and users of functions, modules, or libraries. Modern software Integrated Development Environments (IDEs) and code editors also leverage types as additional sources of information for code completion and documentation.
 
However, as the type systems in modern programming languages become increasingly sophisticated, the task of both authoring and comprehending complex types can become daunting, especially for novice users. For instance, when dealing with types found in Lens (Fig. \ref{fig:lens}), a popular Haskell library, understanding and using them necessitates a significant investment of time.
 

\section{Feature Walkthrough}

\section{Evaluation}

\section{Conclusion and Future Work}