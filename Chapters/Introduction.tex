% Chapter 1

\chapter{Introduction}
\label{chapter1} 

\section{Statically Typed Languages and Type Errors}

Strongly typed programming languages play a crucial role in modern software development by ensuring a higher level of safety, reliability, and maintainability in code. They offer a fundamental advantage in catching errors at compile time, thus preventing many runtime issues that can lead to system failures or vulnerabilities. However, the usability of these languages can be significantly hindered by confusing type errors, which continue to pose a challenge in programming with strong type systems.

Confusing type errors are an enduring usability issue in strongly typed languages. While the type system is designed to provide clarity and prevent errors, it can sometimes introduce its own set of challenges. When a type error occurs, developers are presented with cryptic error messages and often struggle to understand the underlying issue. These messages can be overwhelming, especially for newcomers to the language, leading to frustration and impeding productivity.

The current tools and diagnostics for handling type errors in many strongly typed languages leave much to be desired in terms of usability, design, and practicality. Error messages are often overly technical and lack user-friendly explanations. This can discourage developers from embracing strong typing, or even lead to workarounds that undermine the benefits of the type system. Furthermore, the tools provided for type error diagnosis often lack the sophistication needed to pinpoint the root cause of complex type issues and to offer actionable suggestions for resolution.

To fully harness the advantages of strongly typed languages, there is a pressing need for tools and systems that enhance the usability of type error handling. These tools should be designed with the developer's experience in mind, offering clear, human-readable error messages, actionable insights, and the ability to guide developers through the process of fixing type issues. User-friendly, informative error messages can help developers learn and understand the language better, improving their overall experience and making strong typing more approachable.

Improvements in the design and practicality of type error handling tools are essential for nurturing a thriving community of developers in strongly typed languages. By addressing the usability challenges associated with type errors, these languages can continue to provide the safety and robustness they are known for while becoming more accessible to a broader range of developers.






\section{Research Questions} \label{sec:research-questions}


\subsection{RQ1: How can we make Haskell's type error messages more user-friendly, informative, and easier to understand for developers, especially novice users?}

\subsection{RQ2: How can we develop more sophisticated type error diagnosis techniques that provide precise information about the source of type errors and suggest potential fixes?}


\subsection{RQ3: Can visualization help lower the difficulties of reading, understanding and comparing types?}


\subsection{RQ3: Can we combine the debugging paradigms we explored into a integrated debugging framework?}

\section{Contributions}  \label{sec:contributions}

\subsection{Chameleon: An MUS-based Type Error Diagnosis Tool}


\subsection{Goanna: An MCS-based Type Error Diagnosis Tool}


\subsection{A Diagrammatic Notation for Haskell Types}


\section{Thesis Outline}