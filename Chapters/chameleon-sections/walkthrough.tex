\section{Walkthrough} \label{sec:walkthrough}
In this section, we showcase \chameleon{} by walking through examples of its
use. The examples are given from the perspective of a hypothetical Haskell 
programmer Maxine. 

% \todo{A bit of redundant/repeated here c.f. prev section - if this is Usage example, again use motivating example in it?}

\subsection{Basic mode} \label{sub:basic}
Maxine writes a function to calculate the sum of a list of
numbers, but \chameleon{} shows there is a type error (Fig.~\ref{fig:basic-mode-1}). 
After reading the error reports, Maxine realizes that the error revolves 
around the expression \texttt{xs}. That is: \texttt{xs} can be
either \texttt{[a]} or \texttt{Int}. By matching the color in the
conflicting type block (Fig. \ref{fig:anatomy}-H) and the highlighted error locations 
Maxine knows that the \texttt{[a]} results from the pattern matching of the
\texttt{:} operator, while \texttt{Int} results from using \texttt{+} to
 add two expressions. 

\begin{figure}
        \centering
        \includegraphics[width=\linewidth,trim=0mm 8mm 0mm 0mm]{images/basic-mode-1.pdf}
        \caption{
            Maxine's code to calculate the sum of a list of integers;
            \chameleon{} reports an error on the expression \texttt{xs}.
            }
            \label{fig:basic-mode-1}
\end{figure}

% \begin{figure}[h]
%         \centering
%         \includegraphics[width=\linewidth]{images/basic-mode-2.pdf}
%         \caption{
%        Hovering on possible type 1 will limit the highlights 
%        in the editor to only the ones (in orange) that support \texttt{x::[a]}.
%         }
%         \label{fig:basic-mode-2}
% \end{figure}

% \begin{figure}
%         \centering
%         \includegraphics[width=\linewidth]{images/basic-mode-3.pdf}
%         \caption{
%             Hovering on possible type 2 (Fig. \ref{fig:basic-mode-2}-3) will limit the highlights 
%             in the editor to only the ones (in blue) that support \texttt{x::Int}.
%         }
%         \label{fig:basic-mode-3}
% \end{figure}





At this point, Maxine knows the possible type 1 aligns with her intention, and therefore, the error locations with blue highlights must be erroneous. After examining the program, it comes clear that Maxine forgets to apply the \texttt{sum} function recursively at the right-hand side of the addition. 


\subsection{Balanced mode} \label{sub:balanced}


Maxine writes additional code to add only even numbers in a list of integers, reusing the \texttt{sum} function she wrote earlier. After saving the file, \chameleon{} shows a type error in the expression \texttt{sum} (Fig.~\ref{fig:balance-mode-1}). However, this is not helpful because Maxine has just verified the implementation of \texttt{sum}. Switching to balanced mode, \chameleon{} shows two cards: \texttt{sum} and \texttt{evens}. 

\begin{figure}
        \centering
        \includegraphics[width=\linewidth,trim=0mm 8mm 0mm 0mm]{images/balanced-mode-1.pdf}
        \caption{
            Maxine's code to calculate only the sum 
            of even numbers. \chameleon{} reports 
            an error with two candidate expressions.
        }
        \label{fig:balance-mode-1}
\end{figure}


\begin{figure}
   \centering
        \includegraphics[width=\linewidth,trim=0mm 8mm 0mm 0mm]{images/balanced-mode-2.pdf}
        \caption{
            Clicking on the \texttt{evens} card (5) results in the changes in the
            conflicting types panel to show the possible types for \texttt{evens},
            and the changes highlight color to reflect the assumption that the
            definition of \texttt{evens} is the cause of the error.  
        }
        \label{fig:balance-mode-2}
\end{figure}

Maxine therefore clicks on the \texttt{evens} card and \chameleon{} reports two
possible types for the expression \texttt{[Int]} and \texttt{[Int] -> [Int]}
(Fig. \ref{fig:balance-mode-2}). Knowing the expression \texttt{evens} holds
a temporary list of even integers (hence it is of \texttt{[Int]} types), Maxine
concludes that the Possible type 2 is unintended. The locations with blue highlights must
contain the cause. It does not take long for Maxine to realize the list 
\texttt{l} is not supplied to the \texttt{filter} function.


\subsection{Advanced mode}  \label{sub:advanced}



\begin{figure}
        \centering
        \includegraphics[width=\linewidth,trim=0mm 8mm 0mm 0mm]{images/advanced-mode-1.pdf}
        \caption{
            Maxine's code to calculate only the sum 
            of even numbers in advanced mode. 
            The current step is step 5, \chameleon{} 
            explains that the two appearances of expression 
            \texttt{evens} should have the same type.
        }
        \label{fig:advanced-mode-step5}
\end{figure}

\begin{figure}
        \centering
        \includegraphics[width=\linewidth,trim=0mm 8mm 0mm 0mm]{images/advanced-mode-2.pdf}
        \caption{
            In step 6, \chameleon{} 
            explains that \texttt{evens} is defined as
            the expression \texttt{filter isEven}. The left-hand side
            and the right-hand side should have the same type.
        }
        \label{fig:advanced-mode-step6}
\end{figure}

\begin{figure}
        \centering
        \includegraphics[width=\linewidth,trim=0mm 8mm 0mm 0mm]{images/advanced-mode-3.pdf}
        \caption{
            In step 7, \chameleon{} 
            explains that \texttt{filter} is applied to 
            the function \texttt{isEven}. Assisted by 
            the type of \texttt{filter} in the 
            Relevant Type Information panel on the bottom
            right, Maxine can find the type error that 
            \texttt{filter} expects two arguments but receives one.
        }
        \label{fig:advanced-mode-step7}
\end{figure}

% For the task shown in section \ref{sub:balanced}, if Maxine is not satisfied by
% the options provided by \chameleon{}, by switching to advanced
% mode, she has access to the debugging steps. 

To illustrate the deduction steps with the task shown in section \ref{sub:balanced},  first, Maxine clicks on step 5 (Fig. \ref{fig:advanced-mode-step5}) and verifies
that the two occurrences of \texttt{evens} are supposed to be identical, and the
second use means \texttt{evens} is a list of integers. Second, she
clicks on step 6 (Fig. \ref{fig:advanced-mode-step6}) and verifies that
\texttt{evens} should be the same type as the declaration on the right-hand
side. 


Lastly, Maxine clicks on step 7 (Fig. \ref{fig:advanced-mode-step7}), and
it shows that the \texttt{filter} function is applied to one argument
\texttt{isEven}. By consulting the relevant type information, Maxine identifies
that \texttt{filter} is expecting two arguments while only one is provided. 

